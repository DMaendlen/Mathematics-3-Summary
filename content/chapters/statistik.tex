\chapter{Statistik}
\section{Häufigkeiten}
\begin{table}[h]
	\begin{tabular}{lll}
		Variable & Formel & Bedeutung\\
		\toprule
		$n_j$ & & absolute Häufigkeit\\
		$N$ & $\sum\limits_j n_j$ & Anzahl Messdaten\\
		$h_j$ & $\frac{n_j}{N}$ & relative Häufigkeit, $\sum\limits_j h_j = 1$\\
		$N_j$ & $\sum\limits_{k=1}^{j} n_k$ & absolute Summenhäufigkeit\\
		$H_j$ & $\sum\limits_{k=1}^{j} h_k$ & relative Summenhäufigkeit, $\sum\limits_j H_j = 1$\\
	\end{tabular}
\end{table}

\section{Klassische Wahrscheinlichkeit}
\begin{equation}
P(A) = \frac{|A|}{|\Omega|} = \frac{\text{Anzahl günstiger Fälle}}{\text{Anzahl möglicher Fälle}}
\end{equation}

\section{Permutationen}
Eine Anordnung von $n$ Elementen in einer bestimmten Reihenfolge heißt Permutation.

Für die Anzahl gilt:
\begin{equation}
	P(n) = n!
\end{equation}

Sind $n_1, n_2, n_3, \dots, n_k$ Elemente jeweils einander gleich:\\
\begin{equation}
	P(n; n_1,n_2,\dots,n_k) = \frac{n!}{n_1!\cdot n_2!\dots n_k!}
\end{equation}

\section{Kombinatorik}
\begin{table}[h]
	\begin{tabular}{l|ll}
				& Ohne Zurücklegen & Mit Zurücklegen \\
		\toprule
		geordnet	& $N=\frac{n!}{(n-k)!}$ & $N=n^k$\\
		\midrule
		ungeordnet	& $N=\binom{n}{k}=\frac{n!}{(n-k)!k!}$ & $N=\binom{n+k-1}{k}$
	\end{tabular}
	\caption{Entscheidungstabelle Wahrscheinlichkeiten}
\end{table}

\section{Variablen}
\begin{table}[h]
	\begin{tabular}{lll}
		Variable & Bedeutung & Beispiel\\
		\toprule
		$n$ & Größe der Grundgesamtheit & Kugeln in der Urne\\
		$k$ & Größe der Stichprobe & gezogene Kugeln\\
		$N$ & Anzahl der möglichen Kombinationen & Ergebnis
	\end{tabular}
	\caption{Variablen}
\end{table}

\section{Rechenregeln der Wahrscheinlichkeit}
\begin{align*}
	& P(\emptyset) = \frac{|\emptyset|}{|\Omega|}=0\\
	& P(\Omega) = \frac{|\Omega|}{|\Omega|}=1\\
	& \Rightarrow  0\leq P(A) \leq 1\\
	& P(A\cup B) = P(A)+P(B)-P(A\cap B)\\
	& P(A)+P(\bar{A}) =1\\
	& P(A\cup B) = P(A) + P(B) \Leftrightarrow A\cup B = \emptyset
\end{align*}

\section{Bedingte Wahrscheinlichkeit und Unabhängigkeit}
\begin{equation}
	P(A|B) = \frac{P(A\cap B)}{P(B)}
\end{equation}
\begin{equation}
	P(A\cap B)=P(A|B)\cdot P(B)
\end{equation}
\begin{equation}
	P(A\cap B)=P(B|A)\cdot P(A)
\end{equation}
$\Rightarrow$ Multiplikationssatz:
\begin{equation}
	P(A|B)\cdot P(B) = P(B|A)\cdot P(A)
\end{equation}
\begin{equation}
	P(A\cap B) = P(A)\cdot P(B) \Rightarrow \text{A unabhängig von B}
\end{equation}

\section{Totale Wahrscheinlichkeit}
\subsection{Voraussetzungen}
\begin{equation*}
	B_i \cap B_j = \emptyset\\
\end{equation*}
\begin{equation*}
	B_1 \cup \dots \cup B_n = \Omega
\end{equation*}
\begin{equation*}
	A \subseteq \Omega
\end{equation*}

\subsection{Formel}
\begin{equation}
	P(A) = P(A|B_1)\cdot P(B_1) + P(A|B_2) \cdot P(B_2) + \dots + P(A|B_n)\cdot P(B_n)
\end{equation}

\subsection{Pfadregeln}
Parallel verlaufende Pfade werden addiert, hintereinander folgende Pfade multipliziert.\\
Beispiel:\\
Maschine, 3 Bauteile, 2 davon parallel zueinander, danach das dritte. Die Ausfallwahrscheinlichkeiten der beiden ersten werden addiert, dann mit der Ausfallwahrscheinlichkeit des dritten multipliziert um die Ausfallwahrscheinlichkeit der gesamten Maschine zu berechnen.

\clearpage

\section{Bayes}
\subsection{Voraussetzungen}
\begin{equation*}
	B_i \cap B_j = \emptyset\\
\end{equation*}
\begin{equation*}
	B_1 \cup \dots \cup B_n = \Omega
\end{equation*}
\begin{equation*}
	A \subseteq \Omega
\end{equation*}

\subsection{Formel}
\begin{equation}
	P(B_k|A) = \frac{P(A|B_k)\cdot P(B_k)}{\sum\limits_{k=1}^n P(A|B_k) \cdot P(B_k)}
\end{equation}

\section{Zufallsvariable}
Sei $\omega \in \Omega$ ein Elementarereignis. Dann ist $X(\omega)$ Zufallsvariable. $X(\omega)$ hängt von der Fragestellung ab.

\section{Wahrscheinlichkeitsdichte}
\begin{equation}
	f(x) = P(X = x)
\end{equation}

\section{Diskrete ZV}
Die aufsummierte Wahrscheinlichkeitsdichte heißt Verteilungsfunktion $F(x)$\\
Erwartungswert $\mu$:
\begin{align}
	& \mu = E(X) = \sum_i x_i \cdot f(x_i)\\
	& E(nX) = nE(X)\\
	& E(X_1 + X_2) = E(X_1) + E(X_2)
\end{align}

Fairness:\\
Ist der Erwartungswert $E(X) = 0$ gilt beispielsweise ein Würfelspiel als fair.\\


Varianz $\sigma^2$:
\begin{align}
	& \sigma^2 = \mathrm{Var}(X)=\sum_i(x_i-\mu)^2\cdot f(x_i) = \sum_i x_i^2f(x_i)-\mu^2\\
	& \mathrm{Var}(nX) = n^2\mathrm{Var}(X)\\
	& \mathrm{Var}(nX_1+mX_2) = n^2\mathrm{Var}(X_1) + m^2\mathrm{Var}(X_2)
\end{align}

Standardabweichung $\sigma$:
\begin{equation}
	\sigma = \sqrt{\sigma^2}
\end{equation}

\clearpage

\section{Stetige ZV}
Erwartungswert $\mu$:
\begin{equation}
	\mu = \int_{-\infty}^{\infty}xf(x)\mathrm{d}x\\
\end{equation}
Varianz $\sigma^2$:
\begin{equation}
	\sigma^2 = \int_{-\infty}^{\infty}x^2f(x)\mathrm{d}x-\mu^2\\
\end{equation}
Standardabweichung $\sigma$:
\begin{equation}
	\sigma = \sqrt{\sigma^2}
\end{equation}
\begin{align}
	P(X\leq a) &= F(a)\\
	P(X > a) &= 1 - F(a)\\
	P(a < X < b) &= F(b) - F(a)
\end{align}

Außerdem gilt:
\begin{align}
	F(-\infty) &= 0\\
	F(\infty) &= 1\\
	\int_{-\infty}^\infty\!f(x)\, \mathrm{d}x &= 1
\end{align}

\clearpage

\section{Verteilungen}
\subsection{Binomialverteilung}
\large{Ziehen mit Zurücklegen, Bernoulli-Experiment}\\
Diskrete Dichte:
\begin{equation}
	P(X=k) = \binom{n}{k}p^k(1-p)^{n-k}
\end{equation}

Erwartungswert:
\begin{equation}
	\mu = E(X) = np
\end{equation}

Varianz:
\begin{equation}
	\text{Var}(X)=np(1-p)
\end{equation}

Approximation durch Poisson-Verteilung:
\begin{align*}
	& \text{Faustregel} \\
	& np \leq 10 \text{ und} \\
	& n \geq 1500p \Rightarrow Ps(\mu = np) 
\end{align*}

Approximation durch Normalverteilung:
\begin{align*}
	& \text{Faustregel} \\
	& np(1-p) > 9 \Rightarrow N(\mu = np; \sigma = \sqrt{np(1-p)} 
\end{align*}

Anwendungsbeispiel:\\
Würfelspiel, 3 Würfel, einer davon zeigt die gewünschte Augenzahl:\\
\begin{equation}
	\binom{3}{1}\cdot\left(\frac{1}{6}\right)^1\cdot\left(1-\frac{1}{6}\right)^2
\end{equation}

\begin{table}[h]
	\begin{tabular}{ll}
		Variable & Erklärung\\
		\toprule
		$n$ & Anzahl Würfel\\
		$k$ & Anzahl Treffer (günstiges Ereignis)\\
		$p$ & Wahrscheinlichkeit einer bestimmten Augenzahl
	\end{tabular}
\end{table}

\clearpage

\subsection{Hypergeometrische Verteilung}
\large{Ziehen ohne Zurücklegen}\\
Diskrete Dichte:
\begin{equation}
	P(X=k)=\frac{\binom{M}{k}\binom{N-M}{n-k}}{\binom{N}{n}}
\end{equation}

Erwartungswert:
\begin{equation}
	\mu = E(X)= n\frac{M}{N}
\end{equation}

Varianz:
\begin{equation}
	\mathrm{Var}(X)=n\frac{M}{N}\left(1-\frac{M}{N}\right)\left(\frac{N-n}{N-1}\right)
\end{equation}

Approximation durch Binomialverteilung
\begin{align*}
	& \text{Faustregel} \\
	& 0,1 < \frac{M}{N} < 0,9 \\
	& n < 0,05N, n >10 \Rightarrow B\left(n;p = \frac{M}{N}\right) 
\end{align*}

Approximation durch Poisson-Verteilung
\begin{align*}
	& \text{Faustregel} \\
	& \frac{M}{N}\leq 0,1 \text{ oder } \frac{M}{N}\geq 0,9 \\
	& n < 0,05N, n > 30 \Rightarrow Ps\left(\mu = n\frac{M}{N}\right) 
\end{align*}

Approximation durch Normalverteilung
\begin{align*}
	& \text{Faustregel}  \\
	& 0,1 < \frac{M}{N} < 0,9  \\
	& n < 0,05N, n > 30 \Rightarrow N\left(\mu = n\frac{M}{N}; \sigma = \sqrt{n\frac{M}{N}\left(1-\frac{M}{N}\right)\frac{N-n}{N-1}}\right) 
\end{align*}

Anwendungsbeispiel:\\
N Kugeln, M davon rot, n Kugeln werden gezogen, P ist die Wahrscheinlichkeit, dass von n Kugeln genau k rot sind.

\subsection{Poissonverteilung}
\large{Mehrstufiges Bernoulli-Experiment}\\
Diskrete Dichte:
\begin{equation}
	P(X=k)=\frac{\lambda^k}{k!}\e^{-\lambda}
\end{equation}

Erwartungswert:
\begin{equation}
	\mu = E(X) = \lambda
\end{equation}

Varianz:
\begin{equation}
	\mathrm{Var}(X)=\lambda
\end{equation}

Approximation durch Normalverteilung
\begin{align*}
	& \text{Faustregel} \\
	& \mu > 10 \Rightarrow N(\mu; \sigma = \sqrt{\mu}) 
\end{align*}

Anwendungsbeispiel:\\
Blitze pro Stunde pro $\mathrm{km}^2$, P ist die Wahrscheinlichkeit, dass ein Ereignis k-mal eintritt, wobei es im Mittel $\lambda$-mal eintritt.

\clearpage

\subsection{Normalverteilung}
Dichtefunktion der NV mit $\mu$ und $\sigma^2$:
\begin{equation}
	f(x) = \frac{1}{\sqrt{2\pi\sigma^2}}\e^{-\frac{1}{2}\frac{(x-\mu)^2}{\sigma^2}}
\end{equation}

Verteilungsfunktion:
\begin{equation}
	F(x) = \int_{-\infty}^xf(t)\mathrm{d}t
\end{equation}

Da $f(x)$ sich nicht integrieren lässt, lässt sich $F(x)$ nicht analytisch berechnen. Ausweichen auf die Standardnormalverteilung $\Phi(z)$ samt Tabelle.

Standardnormalverteilung hat $\mu=0$ und $\sigma^2=1$.\\
Dichtefunktion:
\begin{equation}
	\Phi(z) = \frac{1}{\sqrt{2\pi}}\e^{-\frac{1}{2}x^2}
\end{equation}

Berechnung von Wahrscheinlichkeiten mit der NV:\\
\begin{align}
	\begin{rcases}
		X \text{ sei } N(\mu, \sigma^2)\text{-verteilt} &\\
		Z \text{ sei } N(0,1)\text{-verteilt} &
	\end{rcases} \text{Umrechnungsformel: } z=\frac{x-\mu}{\sigma}
\end{align}

\begin{align}
	P(X < a) &= F(a) = \Phi\left(\frac{a-\mu}{\sigma}\right)\\
	P(X > a) &= 1-F(a) = 1-\Phi\left(\frac{a-\mu}{\sigma}\right)\\
	P(a < X < b) &= F(b) - F(a) = \Phi\left(\frac{b-\mu}{\sigma}\right) - \Phi\left(\frac{a-\mu}{\sigma}\right)
\end{align}

\subsection{Quantile}
Quantile der Normalverteilung:\\
$x_{0,95}$ heißt 95\%-Quantil.\\
Es gilt: $P(X < x_{0,95}) = F(X_{0,95}) = 0,95$.\\

Quantile der Standardnormalverteilung:\\
$z_{0,8}, z_{0,95}\dots$ siehe Tabelle.\\

Umrechnung:\\
$x_{0,95} = \mu +\sigma\cdot z_{0,95}$\\

Zu beachten: $\alpha=$Signifikanzniveau (Irrtumswahrscheinlichkeit).\\
Schreibweise: $x_{1-\alpha}=\mu+\sigma\cdot z_{1-\alpha}$

\section{Zufallsstreubereiche}
95\%-ZSB: Intervall, in dem $X$ mit 95\%-iger Wahrscheinlichkeit liegt.

\begin{itemize}
	\item zweiseitiger ZSB: $\left[\mu-z_{1-\frac{\alpha}{2}}\sigma; \mu+z_{1-\frac{\alpha}{2}}\sigma\right]$
	\item einseitiger, nach oben beschränkter ZSB: $(-\infty;\mu+z_{1-\alpha}\sigma]$
	\item einseitiger, nach unten beschränkter ZSB: $[\mu-z_{1-\alpha}\sigma; \infty)$
\end{itemize}

\section{Zufallsstreubereich für $\bar{X}$}

\begin{itemize}
	\item zweiseitiger ZSB: $\left[\mu-z_{1-\frac{\alpha}{2}}\frac{\sigma}{\sqrt{n}}; \mu+z_{1-\frac{\alpha}{2}}\frac{\sigma}{\sqrt{n}}\right]$
	\item einseitiger, nach oben beschränkter ZSB: $(-\infty;\mu+z_{1-\alpha}\frac{\sigma}{\sqrt{n}}]$
	\item einseitiger, nach unten beschränkter ZSB: $[\mu-z_{1-\alpha}\frac{\sigma}{\sqrt{n}}; \infty)$
\end{itemize}

\section{Schließende Statistik}
\subsection{Punktschätzer}
Es liegt eine Messreihe $X_1, X_2, \dots, X_n$ vor.

\begin{table}[ht!]
	\begin{tabular}{ll}
		Unbekannter Parameter & Punktschätzer\\
		\toprule
		Wahrscheinlichkeit $p$ & $\hat{p}=\frac{k}{n}\dots$ relative Häufigkeit\\
		Erwartungswert $\mu$ & $\hat{\mu}=\bar{x}\dots$ arithmetisches Mittel\\
		Varianz $\sigma^2$ & $\hat{\sigma}^2 = s^2\dots$ empirische Varianz\\
		Standardabweichung $\sigma$ & $\hat{\sigma} = s\dots$ empirische Standardabweichung
	\end{tabular}
	\caption{Punktschätzer}
\end{table}

\subsection{Statistische Tests}
Ablauf:
\begin{itemize}
	\item Formulierung einer Nullhypothese $H_0$ und der zugehörigen Alternativhypothese $H_1$
	\item Wahl des Signifikanzniveaus $\alpha$
	\item ZSB wird nach Tabelle je nach Testart berechnet
	\item Testentscheidung, wenn $\bar{x}\notin\mathrm{ZSB}\Rightarrow H_0$ wird verworfen $\Rightarrow H_1$ gilt als bestätigt, mit Irrtumswahrscheinlichkeit $\alpha$, wenn $\bar{x}\in\mathrm{ZSB}\Rightarrow$ keine Aussage möglich
\end{itemize}

\subsection{Konfidenzintervalle}
Konfidenzintervalle bzw. Vertrauensbereiche sind Intervalle, in denen $\mu$ (Fälle A \& B), $\mu_1-\mu_2$ (Fall C) bzw. $p$ (Fall D) mit der Wahrscheinlichkeit $1-\alpha$ liegt. Es gibt zweiseitige, einseitig unten und einseitig oben beschränkte Konfidenzintervalle.

\clearpage

\section{Beschreibende Statistik}

\subsection{Arithmetisches Mittel}
\begin{align*}
\overline{x} = \frac{1}{n} \cdot \sum_{i=1}^{n} x_i
\end{align*}

\subsection{Empirische Varianz}
\begin{align*}
	s^2 &= \frac{1}{n-1} \cdot \sum_{i=1}^{n} \left(x_i - \overline{x}\right)^2\\
	    &= \frac{1}{n-1} \cdot \left(\left(\sum_{i=1}^{n} x_i^2 \right) - n \cdot \overline{x}^2\right)
\end{align*}

\subsection{Empirische Standardabweichung}
\begin{align*}
	s = \sqrt{s^2} = \sqrt{\frac{1}{n-1} \cdot \sum_{i=1}^{n} (x_i - \overline{x})^2} = \sqrt{\frac{1}{n-1} \cdot ((\sum_{i=1}^{n} x_i^2) - n \cdot \overline{x}^2)}
\end{align*}

\subsection{Empirische Kovarianz}
\begin{align*}
	s_{xy} &= \frac{1}{n-1} \cdot \sum_{i=1}^{n} (x_i-\overline{x})(y_i-\overline{y})\\
	&= \frac{1}{n-1} \cdot \left(\sum_{i=1}^{n} (x_i \cdot y_i) - n \cdot \overline{x} \cdot \overline{y}\right)\\
\end{align*}

Anwendungsbeispiel: Kovarianz von Gewicht und Haltbarkeit von Werkzeugen\\
Ist das Vorzeichen negativ nimmt die Haltbarkeit mit steigendem Gewicht des Werkzeugs ab, ist es positiv nimmt sie zu.

\subsection{Empirischer Korrelationskoeffizient}
\begin{align*}
	r_{xy} = \frac{s_{xy}}{s_x \cdot s_y}\\
	-1 \leq r_{xy} \leq 1
\end{align*}

Für $|r_{xy}|\rightarrow 1$ gilt starke Abhängigkeit von $x$ und $y$, für $|r_{xy}|\rightarrow 0$ ist die Abhängigkeit schwach.

\subsection{Regressionsgerade}
\begin{align*}
	Q(m,b) &= \sum_{i=1}^{n}\left[y_i-(mx_i+b)\right]^2=\text{Min!}\\
	m &= \frac{s_{xy}}{s_x^2}\\
	b &= \bar{y}-m\cdot\bar{x}
\end{align*}

Alternative Formel:

\begin{align*}
	y &= ax+b\\
\\
	a &= \frac{n \sum\limits_{i=1}^{n} x_i y_i - \left( \sum\limits_{i=1}^{n} x_i \right) \left( \sum\limits_{i=1}^{n} y_i \right)}
		{n \sum\limits_{i=1}^{n} x_i^2 - \left( \sum\limits_{i=1}^{n} x_i \right)^2}\\
\\
	b &= \frac{\left( \sum\limits_{i=1}^{n} x_i^2 \right) \left( \sum\limits_{i=1}^{n} y_i \right)-
		\left( \sum\limits_{i=1}^{n} x_i \right) \left( \sum\limits_{i=1}^{n} x_i y_i \right)}
		{n \sum\limits_{i=1}^{n} x_i^2- \left( \sum\limits_{i=1}^{n} x_i \right)^2}
\end{align*}

\subsection{Median}
Der Median ist die exakte Mitte einer geordneten Messreihe.

\begin{equation*}
	\tilde{x} = x_{0,5}
\end{equation*}


\clearpage

\section{Statistische Tests}

\subsection{Test für den Parameter $\mu$ einer Normalverteilung bei bekannter Varianz (Gauß-Test)}
Stichprobe $x_1, \ldots , x_n$\\
Messwerte sind Realisierungen von $n$ unabhängigen $N(\mu, \sigma^2)$ verteilten Zufallsvariablen mit unbekanntem Erwartungswert $\mu$, aber bekannter Varianz $\sigma^2$.\\
Signifikanzniveau $\alpha$\\

\begin{tabular}{llll}
	$H_0$ 		& $H_1$ 		& Zufallsstreubereich (ZSB) für $\overline{x}$, falls $H_0$ zutrifft & $H_0$ verwerfen falls\\
	\toprule
	$\mu = \mu_0$ 	& $\mu \ne \mu_0$ 	& $[\mu_0 - z_{1-\alpha /2} \cdot \frac{\sigma}{\sqrt{n}} ; \;
	\mu_0 + z_{1-\alpha /2} \cdot \frac{\sigma}{\sqrt{n}}]$  & $\overline{x} \not\in$ ZSB \\
	$\mu \ge \mu_0$ & $\mu < \mu_0$ & $[\mu_0 - z_{1-\alpha} \cdot \frac{\sigma}{\sqrt{n}} ; \;
	\infty)$ & $\overline{x} \not\in$ ZSB \\
	$\mu \le \mu_0$ & $\mu > \mu_0$ & $(-\infty ; \;
	\mu_0 + z_{1-\alpha} \cdot \frac{\sigma}{\sqrt{n}}]$  & $\overline{x} \not\in$ ZSB 
\end{tabular}


\subsection{Test für den Parameter $\mu$ einer Normalverteilung bei unbekannter Varianz ($t$-Test)}

Stichprobe $x_1, \ldots , x_n$\\
Messwerte sind Realisierungen von $n$ unabhängigen $N(\mu, \sigma^2)$ verteilten Zufallsvariablen mit unbekanntem Erwartungswert $\mu$ und unbekannter Varianz $\sigma^2$.\\
Die unbekannte Standardabweichung $\sigma$ wird durch die empirische Standardabweichung $s$ aus der Stichprobe geschätzt.\\
Signifikanzniveau $\alpha$\\


\begin{tabular}{llll}
	$H_0$ 		& $H_1$ 	& Zufallsstreubereich (ZSB) für $\overline{x}$, falls $H_0$ zutrifft & $H_0$ verwerfen falls\\
	\toprule
	$\mu = \mu_0$ 	& $\mu \ne \mu_0$ & $[\mu_0 - t_{n-1;1-\alpha /2} \cdot \frac{s}{\sqrt{n}} ; \;
	\mu_0 + t_{n-1;1-\alpha /2} \cdot \frac{s}{\sqrt{n}}]$  & $\overline{x} \not\in$ ZSB \\ 
	$\mu \ge \mu_0$ & $\mu < \mu_0$ & $[\mu_0 - t_{n-1;1-\alpha} \cdot \frac{s}{\sqrt{n}} ; \;
	\infty)$ & $\overline{x} \not\in$ ZSB \\
	$\mu \le \mu_0$ & $\mu > \mu_0$ & $(-\infty ; \;
	\mu_0 + t_{n-1;1-\alpha} \cdot \frac{s}{\sqrt{n}}]$  & $\overline{x} \not\in$ ZSB 
\end{tabular}

\clearpage

\subsection{Vergleich der Parameter $\mu_1$ und $\mu_2$ zweier Normalverteilungen (Zweistichproben-$t$-Test)}

Zwei Stichproben $x_1, \ldots , x_m$ und $y_1, \ldots , y_n$\\
Die $m$ Messwerte $x_1, \ldots , x_m$ sind Realisierungen von $N(\mu_1, \sigma^2)$ verteilten Zufallsvariablen; die $n$ Messwerte $y_1, \ldots , y_n$ sind Realisierungen von $N(\mu_2, \sigma^2)$ verteilten Zufallsvariablen. Alle Zufallsvariablen sind voneinander unabhängig mit der gleichen unbekannten Varianz $\sigma^2$.\\
Aus den beiden empirischen Varianzen $s_x^2$ der ersten Stichprobe und $s_y^2$ der zweiten Stichprobe muss zunächst die folgende Hilfsgröße berechnet werden:
\begin{align*}
s_d = \sqrt{(m-1) s_{x}^2 + (n-1) s_{y}^2} \cdot \sqrt{\frac{m+n}{m \cdot n \cdot (m+n-2)}}
\end{align*}
Signifikanzniveau $\alpha$\\

\begin{tabular}{llll}
	$H_0$ 		& $H_1$ 	& Zufallsstreubereich (ZSB) für $\overline{x} - \overline{y}$, falls $H_0$ zutrifft & $H_0$ verwerfen falls \\
	\toprule
	$\mu_1 = \mu_2$ & $\mu_1 \ne \mu_2$ & $[- t_{m+n-2;1-\alpha /2} \cdot s_d ; \;
	t_{m+n-2;1-\alpha /2} \cdot s_d]$  & $\overline{x}-\overline{y} \not\in$ ZSB \\
	$\mu_1 \ge \mu_2$ & $\mu_1 < \mu_2$ & $[- t_{m+n-2;1-\alpha} \cdot s_d ; \; \infty)$  & $\overline{x}-\overline{y} \not\in$ ZSB \\
	$\mu_1 \le \mu_2$ & $\mu_1 > \mu_2$ & $(-\infty ; \;
	t_{m+n-2;1-\alpha} \cdot s_d]$  & $\overline{x}-\overline{y} \not\in$ ZSB
\end{tabular}

\subsection{Test für den Parameter $p$ einer Binomialverteilung}

Hier wird ein Test für ein Ereignis, das mit unbekannter Wahrscheinlichkeit $p$ auftritt, durchgeführt.\\
Hierbei ist $q_0 = 1-p_0$.\\
Signifikanzniveau $\alpha$\\

\begin{tabular}{llll} 
	$H_0$ 		& $H_1$ 	& Zufallsstreubereich (ZSB) für $k/n$, 	falls $H_0$ zutrifft & $H_0$ verwerfen falls \\
	\toprule
	$p=p_0$ 	& $p \ne p_0$ 	& $[p_0 - z_{1-\alpha/2} \cdot \sqrt{\frac{p_0 q_0}{n}} - \frac{0,5}{n}; \; p_0 + z_{1-\alpha/2} \cdot \sqrt{\frac{p_0 q_0}{n}} + \frac{0,5}{n}]$  & $\frac{k}{n} \not\in$ ZSB \\
	$p \ge p_0$ 	& $p < p_0$ 	& $[p_0 - z_{1-\alpha} \cdot \sqrt{\frac{p_0 q_0}{n}} - \frac{0,5}{n}; \; 1 ]$  & $\frac{k}{n} \not\in$ ZSB\\
	$p \le p_0$ & $p > p_0$ & $[0 ; \; p_0 + z_{1-\alpha} \cdot \sqrt{\frac{p_0 q_0}{n}} + \frac{0,5}{n}]$  & $\frac{k}{n} \not\in$ ZSB
\end{tabular}


\clearpage

\section{Tabellen für Vertrauensbereiche (auch: Konfidenzintervalle)}

\subsection{Vertrauensbereiche für den Erwartungswert $\mu$ einer Normalverteilung bei bekannter Varianz $\sigma^2$}

Die Voraussetzungen sind wie im Abschnitt A. bei den statistischen Tests.\\

Der Vertrauensbereich wird auf Basis einer Stichprobe $x_1, \ldots , x_n$ berechnet, deren arithmetischer Mittelwert $\overline x$ ist.\\

\begin{tabular}{ll} 
Art des Vertrauensbereichs & Vertrauensbereich für $\mu$ \\
	\toprule
zweiseitig & $[\overline x - z_{1-\alpha /2} \cdot \frac{\sigma}{\sqrt{n}} ; \;
\overline x + z_{1-\alpha /2} \cdot \frac{\sigma}{\sqrt{n}}]$ \\
einseitig, nach unten begrenzt & $[\overline x - z_{1-\alpha} \cdot \frac{\sigma}{\sqrt{n}} ; \;
\infty)$ \\
einseitig, nach oben begrenzt & $(-\infty ; \; \overline x + z_{1-\alpha} \cdot \frac{\sigma}{\sqrt{n}}]$ 
\end{tabular}

\subsection{Vertrauensbereiche für den Erwartungswert $\mu$ einer Normalverteilung bei unbekannter Varianz}

Die Voraussetzungen sind wie im Abschnitt B. bei den statistischen Tests.\\

Der Vertrauensbereich wird auf Basis einer Stichprobe $x_1, \ldots , x_n$ berechnet, deren arithmetischer Mittelwert $\overline x$ und deren empirische Standardabweichung $s$ ist.\\

\begin{tabular}{ll}
Art des Vertrauensbereichs & Vertrauensbereich für $\mu$ \\
	\toprule
zweiseitig & $[\overline x - t_{n-1;1-\alpha /2} \cdot \frac{s}{\sqrt{n}} ; \;
\overline x + t_{n-1;1-\alpha /2} \cdot \frac{s}{\sqrt{n}}]$ \\
einseitig, nach unten begrenzt & $[\overline x - t_{n-1;1-\alpha} \cdot \frac{s}{\sqrt{n}} ; \;
\infty)$ \\
einseitig, nach oben begrenzt & $(-\infty ; \;
\overline x + t_{n-1;1-\alpha} \cdot \frac{s}{\sqrt{n}}]$ 
\end{tabular}

\clearpage

\subsection{Vertrauensbereiche für die Differenz $\mu_1 - \mu_2$ der Erwartungswerte zweier Normalverteilungen}

Die Voraussetzungen sind wie im Abschnitt C. bei den statistischen Tests. Der Vertrauensbereich wird auf Basis zweier Stichproben $x_1, \ldots , x_m$ und $y_1, \ldots , y_n$ berechnet.\\

Der arithmetische Mittelwert der ersten Stichprobe ist $\overline x$, der der zweiten ist $\overline y$.

Aus den beiden empirischen Varianzen $s_x^2$ der ersten Stichprobe und $s_y^2$ der zweiten Stichprobe muss zunächst die folgende Hilfsgröße berechnet werden:

\begin{align*}
s_d = \sqrt{(m-1) s_{x}^2 + (n-1) s_{y}^2} \cdot \sqrt{\frac{m+n}{m \cdot n \cdot (m+n-2)}}
\end{align*}



\begin{tabular}{ll}
Art des Vertrauensbereichs & Vertrauensbereich für $\mu_1 - \mu_2$ \\
	\toprule
zweiseitig & $[\overline x - \overline y - t_{m+n-2;1-\alpha /2} \cdot s_d ; \;
\overline x - \overline y + t_{m+n-2;1-\alpha /2} \cdot s_d]$ \\
einseitig, nach unten begrenzt & $[\overline x - \overline y - t_{m+n-2;1-\alpha} \cdot s_d ; \; \infty)$ \\
einseitig, nach oben begrenzt  & $(-\infty ; \;
\overline x - \overline y + t_{m+n-2;1-\alpha} \cdot s_d]$ 
\end{tabular}

\subsection{Vertrauensbereiche für eine Wahrscheinlichkeit $p \;$}

Die Voraussetzungen sind wie im Abschnitt D. bei den statistischen Tests. 

Der Vertrauensbereich wird auf Basis einer Stichprobe vom Umfang $n$ berechnet, bei der das gesuchte Ereignis $k$-mal eingetreten sei, Dann ist $\hat p = \frac{k}{n}$ eine Punktschätzung für die unbekannte Wahrscheinlichkeit. Dieser Schätzwert wird für den Vertrauensbereich benutzt.\\

\begin{tabular}{ll}
Art des Vertrauensbereichs & Vertrauensbereich für $p$ \\
	\toprule
zweiseitig & $[\hat p - z_{1-\alpha/2} \cdot \sqrt{\frac{\hat p \cdot (1- \hat p)}{n}} - \frac{0,5}{n}; \; \hat p + z_{1-\alpha/2} \cdot \sqrt{\frac{\hat p \cdot (1- \hat p)}{n}} + \frac{0,5}{n}]$ \\
einseitig, nach unten begrenzt & $[\hat p - z_{1-\alpha} \cdot \sqrt{\frac{\hat p \cdot (1- \hat p)}{n}} - \frac{0,5}{n}; \; 1 ]$ \\
einseitig, nach oben begrenzt & $[0 ; \; \hat p + z_{1-\alpha} \cdot \sqrt{\frac{\hat p \cdot (1- \hat p)}{n}} + \frac{0,5}{n}]$ 
\end{tabular}


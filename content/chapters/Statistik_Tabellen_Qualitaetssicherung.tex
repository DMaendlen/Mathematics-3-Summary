\section{Tabellen für das Kapitel Qualitätssicherung} 

\subsection{Tabelle für die Berechnung der Eingriffsgrenzen}\\

\begin{tabular}{|r|cc|cc|}
\hline
{} & $s$-Karte & {} & $R$-Karte & {}\\
$n$ & $a$ & $b$ & $c$ & $d$\\ \hline
2 & 0,006 & 2,807 & 0,009 & 3,970\\
3 & 0,071 & 2,302 & 0,135 & 4,424\\
4 & 0,155 & 2,069 & 0,343 & 4,694\\
5 & 0,227 & 1,927 & 0,555 & 4,886\\ \hline
6 & 0,287 & 1,830 & 0,749 & 5,033\\
7 & 0,336 & 1,758 & 0,922 & 5,154\\
8 & 0,376 & 1,702 & 1,075 & 5,255\\
9 & 0,410 & 1,657 & 1,212 & 5,341\\
10 & 0,439 & 1,619 & 1,335 & 5,418\\ \hline
\end{tabular}

Hierbei ist $n$ der Umfang der Stichproben, die bei Anwendung der QRK regelmäßig zu entnehmen sind.

\subsection{Tabelle zur Schätzung der Standardabweichung bei der Grunderhebung}

\begin{tabular}{|r|cc|}
\hline
$m$ & $a_m$ & $d_m$\\ \hline
2 & 0,798 & 1,128\\
3 & 0,886 & 1,693\\
4 & 0,921 & 2,059\\
5 & 0,940 & 2,326\\ \hline
6 & 0,952 & 2,534\\
7 & 0,959 & 2,704\\
8 & 0,965 & 2,847\\
9 & 0,969 & 2,970\\
10 & 0,973 & 3,078\\ \hline
\end{tabular}

Hierbei ist $m$ der Umfang der Messreihen bei der Grunderhebung.\\

Für $m>10$ kann man $\sigma$ durch $\hat \sigma \approx \overline s$ schätzen.


\end{itemize}

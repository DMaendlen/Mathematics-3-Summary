\noindent
{\bf \slshape \sffamily Verteilungsfunktion $\Phi(z)$ der Standard-Normalverteilung $N(0;1)$:}\\

\begin{table}[h]
	\small
\begin{tabular}{|c|ccccc|ccccc|}
\hline
z & 0 & 1 & 2 & 3 & 4 & 5 & 6 & 7 & 8 & 9 \\ \hline
0,0 & 0,5000 & 0,5040 & 0,5080 & 0,5120 & 0,5160 & 0,5199 & 0,5239 & 0,5279 & 0,5319 & 0,5359 \\
0,1 & 0,5398 & 0,5438 & 0,5478 & 0,5517 & 0,5557 & 0,5596 & 0,5636 & 0,5675 & 0,5714 & 0,5753 \\
0,2 & 0,5793 & 0,5832 & 0,5871 & 0,5910 & 0,5984 & 0,5987 & 0,6026 & 0,6064 & 0,6103 & 0,6141 \\
0,3 & 0,6179 & 0,6217 & 0,6255 & 0,6293 & 0,6331 & 0,6368 & 0,6406 & 0,6443 & 0,6480 & 0,6517 \\
0,4 & 0,6554 & 0,6591 & 0,6628 & 0,6664 & 0,6700 & 0,6736 & 0,6772 & 0,6808 & 0,6844 & 0,6879 \\
0,5 & 0,6915 & 0,6950 & 0,6985 & 0,7019 & 0,7054 & 0,7088 & 0,7123 & 0,7157 & 0,7190 & 0,7224 \\ \hline
0,6 & 0,7257 & 0,7291 & 0,7324 & 0,7357 & 0,7389 & 0,7422 & 0,7454 & 0,7486 & 0,7517 & 0,7549 \\
0,7 & 0,7580 & 0,7611 & 0,7642 & 0,7673 & 0,7704 & 0,7734 & 0,7764 & 0,7794 & 0,7823 & 0,7852 \\
0,8 & 0,7881 & 0,7910 & 0,7939 & 0,7967 & 0,7995 & 0,8023 & 0,8051 & 0,8078 & 0,8106 & 0,8133 \\
0,9 & 0,8159 & 0,8186 & 0,8212 & 0,8238 & 0,8264 & 0,8289 & 0,8315 & 0,8340 & 0,8365 & 0,8389 \\
1,0 & 0,8413 & 0,8438 & 0,8461 & 0,8485 & 0,8508 & 0,8531 & 0,8554 & 0,8577 & 0,8599 & 0,8621 \\ \hline
1,1 & 0,8643 & 0,8665 & 0,8686 & 0,8708 & 0,8729 & 0,8749 & 0,8770 & 0,8790 & 0,8810 & 0,8830 \\
1,2 & 0,8849 & 0,8869 & 0,8888 & 0,8907 & 0,8925 & 0,8944 & 0,8962 & 0,8980 & 0,8997 & 0,9015 \\
1,3 & 0,9032 & 0,9049 & 0,9066 & 0,9082 & 0,9099 & 0,9115 & 0,9131 & 0,9147 & 0,9162 & 0,9177 \\
1,4 & 0,9192 & 0,9207 & 0,9222 & 0,9236 & 0,9251 & 0,9265 & 0,9279 & 0,9292 & 0,9306 & 0,9319 \\
1,5 & 0,9332 & 0,9345 & 0,9357 & 0,9370 & 0,9382 & 0,9394 & 0,9406 & 0,9418 & 0,9429 & 0,9441 \\ \hline
1,6 & 0,9452 & 0,9463 & 0,9474 & 0,9484 & 0,9495 & 0,9505 & 0,9515 & 0,9525 & 0,9535 & 0,9545 \\
1,7 & 0,9554 & 0,9564 & 0,9573 & 0,9582 & 0,9591 & 0,9599 & 0,9608 & 0,9616 & 0,9625 & 0,9633 \\
1,8 & 0,9641 & 0,9649 & 0,9656 & 0,9664 & 0,9671 & 0,9678 & 0,9686 & 0,9693 & 0,9699 & 0,9706 \\
1,9 & 0,9713 & 0,9719 & 0,9726 & 0,9732 & 0,9738 & 0,9744 & 0,9750 & 0,9756 & 0,9761 & 0,9767 \\
2,0 & 0,9772 & 0,9778 & 0,9783 & 0,9788 & 0,9793 & 0,9798 & 0,9803 & 0,9808 & 0,9812 & 0,9817 \\ \hline
2,1 & 0,9821 & 0,9826 & 0,9830 & 0,9834 & 0,9838 & 0,9842 & 0,9846 & 0,9850 & 0,9854 & 0,9857 \\
2,2 & 0,9861 & 0,9864 & 0,9868 & 0,9871 & 0,9875 & 0,9878 & 0,9881 & 0,9884 & 0,9887 & 0,9890 \\
2,3 & 0,9893 & 0,9896 & 0,9898 & 0,9901 & 0,9904 & 0,9906 & 0,9909 & 0,9911 & 0,9913 & 0,9916 \\
2,4 & 0,9918 & 0,9920 & 0,9922 & 0,9925 & 0,9927 & 0,9929 & 0,9931 & 0,9932 & 0,9934 & 0,9936 \\
2,5 & 0,9938 & 0,9940 & 0,9941 & 0,9943 & 0,9945 & 0,9946 & 0,9948 & 0,9949 & 0,9951 & 0,9952 \\ \hline
2,6 & 0,9953 & 0,9955 & 0,9956 & 0,9957 & 0,9959 & 0,9960 & 0,9961 & 0,9962 & 0,9963 & 0,9964 \\
2,7 & 0,9965 & 0,9966 & 0,9967 & 0,9968 & 0,9969 & 0,9970 & 0,9971 & 0,9972 & 0,9973 & 0,9974 \\
2,8 & 0,9974 & 0,9975 & 0,9976 & 0,9977 & 0,9977 & 0,9978 & 0,9979 & 0,9979 & 0,9980 & 0,9981 \\
2,9 & 0,9981 & 0,9982 & 0,9982 & 0,9983 & 0,9984 & 0,9984 & 0,9985 & 0,9985 & 0,9986 & 0,9986 \\
3,0 & 0,9987 & 0,9987 & 0,9987 & 0,9988 & 0,9988 & 0,9989 & 0,9989 & 0,9989 & 0,9990 & 0,9990 \\ \hline
\end{tabular}
\end{table}

\noindent
Ablesebeispiel: $\Phi(0,92) = 0,8212$\\

\noindent
Werte für negatives $z$ mit der Formel $\Phi(-z) = 1-\Phi(z)$, z.B. $\Phi(-1,55) = 1-0,9394 = 0,0606$

\begin{table}[h]
\begin{tabular}{ll}
	gedämpfte Schwingung & Eine Schwingung ist gedämpft, wenn sie mit $\e^{-x}$ multipliziert wird\\
	$\omega$ & Elementarereignis\\
	$A$  & Ausgang, (günstiges) Ereignis, der gesuchte Wert\\
	$N$  & Anzahl von Möglichkeiten\\
	$|A|$& Anzahl günstiger Fälle / Häufigkeit\\
	$|\Omega|$ & Anzahl möglicher Fälle / Summe der Häufigkeiten\\
	$P(A)=\frac{|A|}{|\Omega|}$ & Wahrscheinlichkeit Für Ausgang / Ereignis A\\
	$f(x_i)$ & Wahrscheinlichkeit / Diskrete Dichte / $P(X=x)$\\
	$F(x)$ & Verteilungsfunktion, aufsummierte W.-Dichte\\
	$\mu$ & arithmetisches Mittel d. W\\
	$\sigma^2=Var(X)$ & Varianz\\
	$\sigma=\sqrt{\sigma^2}$ & Standardabweichung\\
	$X(\omega)$ & Zufallsvariable, Anzahl d. Treffer pro $\omega$\\
	& (wie oft kommt Zahl in 5 Münzwurfen)\\
	$\Omega$ & Anz. d. Durchgänge für $\omega$\\
	& (wie oft werden 5 Münzen geworfen?) \\
	$\alpha$ & Signifikanzniveau (Unsicherheit)\\
	$\Phi$ & Standardnormalverteilung\\
	Regressionsgerade & Beträge der Abstände der Messwerte von der Gerade\\
	Zufallsexperiment & Ausgang offen und zufällig\\
	$\binom{n}{k}$ & Binomialkoeffizient\\
	BV & Binomialverteilung\\
	$\hat{p}=\frac{k}{n}$ & Punktschätzer für die Wahrscheinlichkeit\\
	$\notin$ & nicht Element von \\ 	
	$s^2$ & empirische Standardabweichung\\
	$s$ & empirische Varianz\\
	$\mu$ & Erwartungswert / Sollwert\\
	$\Phi$ & Standardnormalverteilung\\
	$\Phi\left( \frac{\alpha-\mu}{\sigma} \right)$ &  \\
	$f(x)$  & Wahrscheinlichkeitsdichtefunktion\\
	$F(x)$ & Verteilungsfunktion\\
	$x_{0,95}$ & 95\%-Quantil (der Normalverteilung)\\
	$P(X < x_{0,95}) = F(x_{0,95}) = 0,95$ &\\
	ZSB & Zufallsstreubereich\\
\end{tabular}
\end{table}

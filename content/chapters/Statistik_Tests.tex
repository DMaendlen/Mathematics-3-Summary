\section{Statistische Tests}

\subsection{Test für den Parameter $\mu$ einer Normalverteilung bei bekannter Varianz (Gauß-Test)}
Stichprobe $x_1, \ldots , x_n$\\
Messwerte sind Realisierungen von $n$ unabhängigen $N(\mu, \sigma^2)$ verteilten Zufallsvariablen mit unbekanntem Erwartungswert $\mu$, aber bekannter Varianz $\sigma^2$.\\
Signifikanzniveau $\alpha$\\

\begin{tabular}{llll}
	$H_0$ 		& $H_1$ 		& Zufallsstreubereich (ZSB) für $\overline{x}$, falls $H_0$ zutrifft & $H_0$ verwerfen falls\\
	\toprule
	$\mu = \mu_0$ 	& $\mu \ne \mu_0$ 	& $[\mu_0 - z_{1-\alpha /2} \cdot \frac{\sigma}{\sqrt{n}} ; \;
	\mu_0 + z_{1-\alpha /2} \cdot \frac{\sigma}{\sqrt{n}}]$  & $\overline{x} \not\in$ ZSB \\
	$\mu \ge \mu_0$ & $\mu < \mu_0$ & $[\mu_0 - z_{1-\alpha} \cdot \frac{\sigma}{\sqrt{n}} ; \;
	\infty)$ & $\overline{x} \not\in$ ZSB \\
	$\mu \le \mu_0$ & $\mu > \mu_0$ & $(-\infty ; \;
	\mu_0 + z_{1-\alpha} \cdot \frac{\sigma}{\sqrt{n}}]$  & $\overline{x} \not\in$ ZSB 
\end{tabular}


\subsection{Test für den Parameter $\mu$ einer Normalverteilung bei unbekannter Varianz ($t$-Test)}

Stichprobe $x_1, \ldots , x_n$\\
Messwerte sind Realisierungen von $n$ unabhängigen $N(\mu, \sigma^2)$ verteilten Zufallsvariablen mit unbekanntem Erwartungswert $\mu$ und unbekannter Varianz $\sigma^2$.\\
Die unbekannte Standardabweichung $\sigma$ wird durch die empirische Standardabweichung $s$ aus der Stichprobe geschätzt.\\
Signifikanzniveau $\alpha$\\


\begin{tabular}{llll}
	$H_0$ 		& $H_1$ 	& Zufallsstreubereich (ZSB) für $\overline{x}$, falls $H_0$ zutrifft & $H_0$ verwerfen falls\\
	\toprule
	$\mu = \mu_0$ 	& $\mu \ne \mu_0$ & $[\mu_0 - t_{n-1;1-\alpha /2} \cdot \frac{s}{\sqrt{n}} ; \;
	\mu_0 + t_{n-1;1-\alpha /2} \cdot \frac{s}{\sqrt{n}}]$  & $\overline{x} \not\in$ ZSB \\ 
	$\mu \ge \mu_0$ & $\mu < \mu_0$ & $[\mu_0 - t_{n-1;1-\alpha} \cdot \frac{s}{\sqrt{n}} ; \;
	\infty)$ & $\overline{x} \not\in$ ZSB \\
	$\mu \le \mu_0$ & $\mu > \mu_0$ & $(-\infty ; \;
	\mu_0 + t_{n-1;1-\alpha} \cdot \frac{s}{\sqrt{n}}]$  & $\overline{x} \not\in$ ZSB 
\end{tabular}

\clearpage

\subsection{Vergleich der Parameter $\mu_1$ und $\mu_2$ zweier Normalverteilungen (Zweistichproben-$t$-Test)}

Zwei Stichproben $x_1, \ldots , x_m$ und $y_1, \ldots , y_n$\\
Die $m$ Messwerte $x_1, \ldots , x_m$ sind Realisierungen von $N(\mu_1, \sigma^2)$ verteilten Zufallsvariablen; die $n$ Messwerte $y_1, \ldots , y_n$ sind Realisierungen von $N(\mu_2, \sigma^2)$ verteilten Zufallsvariablen. Alle Zufallsvariablen sind voneinander unabhängig mit der gleichen unbekannten Varianz $\sigma^2$.\\
Aus den beiden empirischen Varianzen $s_x^2$ der ersten Stichprobe und $s_y^2$ der zweiten Stichprobe muss zunächst die folgende Hilfsgröße berechnet werden:
\begin{align*}
s_d = \sqrt{(m-1) s_{x}^2 + (n-1) s_{y}^2} \cdot \sqrt{\frac{m+n}{m \cdot n \cdot (m+n-2)}}
\end{align*}
Signifikanzniveau $\alpha$\\

\begin{tabular}{llll}
	$H_0$ 		& $H_1$ 	& Zufallsstreubereich (ZSB) für $\overline{x} - \overline{y}$, falls $H_0$ zutrifft & $H_0$ verwerfen falls \\
	\toprule
	$\mu_1 = \mu_2$ & $\mu_1 \ne \mu_2$ & $[- t_{m+n-2;1-\alpha /2} \cdot s_d ; \;
	t_{m+n-2;1-\alpha /2} \cdot s_d]$  & $\overline{x}-\overline{y} \not\in$ ZSB \\
	$\mu_1 \ge \mu_2$ & $\mu_1 < \mu_2$ & $[- t_{m+n-2;1-\alpha} \cdot s_d ; \; \infty)$  & $\overline{x}-\overline{y} \not\in$ ZSB \\
	$\mu_1 \le \mu_2$ & $\mu_1 > \mu_2$ & $(-\infty ; \;
	t_{m+n-2;1-\alpha} \cdot s_d]$  & $\overline{x}-\overline{y} \not\in$ ZSB
\end{tabular}

\subsection{Test für den Parameter $p$ einer Binomialverteilung}

Hier wird ein Test für ein Ereignis, das mit unbekannter Wahrscheinlichkeit $p$ auftritt, durchgeführt.\\
Hierbei ist $q_0 = 1-p_0$.\\
Signifikanzniveau $\alpha$\\

\begin{tabular}{llll} 
	$H_0$ 		& $H_1$ 	& Zufallsstreubereich (ZSB) für $k/n$, 	falls $H_0$ zutrifft & $H_0$ verwerfen falls \\
	\toprule
	$p=p_0$ 	& $p \ne p_0$ 	& $[p_0 - z_{1-\alpha/2} \cdot \sqrt{\frac{p_0 q_0}{n}} - \frac{0,5}{n}; \; p_0 + z_{1-\alpha/2} \cdot \sqrt{\frac{p_0 q_0}{n}} + \frac{0,5}{n}]$  & $\frac{k}{n} \not\in$ ZSB \\
	$p \ge p_0$ 	& $p < p_0$ 	& $[p_0 - z_{1-\alpha} \cdot \sqrt{\frac{p_0 q_0}{n}} - \frac{0,5}{n}; \; 1 ]$  & $\frac{k}{n} \not\in$ ZSB\\
	$p \le p_0$ & $p > p_0$ & $[0 ; \; p_0 + z_{1-\alpha} \cdot \sqrt{\frac{p_0 q_0}{n}} + \frac{0,5}{n}]$  & $\frac{k}{n} \not\in$ ZSB
\end{tabular}

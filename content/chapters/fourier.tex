\chapter{Fourier-Transformation}
\section{Grundlagen}
\subsection{Heaviside-Funktion $\sigma(t)$}
\subsubsection{Definition}
\begin{align}
	\sigma(t) = \begin{cases}
				1, & \text{für } t > 0\\
				0, & \text{für } t < 0\\
			\end{cases}\\
	\sigma(0) = \text{beliebig} \nonumber
\end{align}

\subsubsection{Verwendung}
\begin{align}
	&\text{Einschalten}\\\nonumber
	&f(t)\cdot\sigma(t-t_0)\\
	&\text{Ausschalten}\\\nonumber
	&f(t)\cdot\left[1-\sigma(t-t_0)\right]\\
	&\text{Ein- und Ausschalten}\\\nonumber
	&f(t)\cdot\left[\sigma(t-t_0) - \sigma(t-t_1)\right]
\end{align}

\subsection{$\delta$-Distribution}

\subsubsection{Definition}
\begin{align}
	\delta(t) = \begin{cases}
				0, & \text{für } t \neq 0\\
				\infty, & \text{für } t = 0\\
			\end{cases}\\
	\int_{-\infty}^\infty\!\delta(t)\,\mathrm{d}t = 1 \nonumber
\end{align}

\subsubsection{Zusammenhang $\sigma(t)$ und $\delta(t)$}
\begin{equation}
	\dot{\sigma}(t) = \delta(t)
\end{equation}

\section{Grundlagen Fourierreihen}
\subsection[Fourierreihe reell mit Periode $P=2\pi$]{Reelle Darstellung der Fourierreihe mit Periode $P=2 \pi$}
\begin{align*}
\begin{array}{rcll}
f(x) & = & \frac{a_0}{2} + \sum_{k=1}^{\infty} \; (a_k \cos(k x) + b_k \sin(k x)) & {}\\
\mbox{mit} & {} & {} & {}\\
	a_k & = & \frac{1}{\pi} \int_{-\pi}^{\pi} \; f(x) \cos(k x) \mathrm{d}x & \qquad k \in \{1,2,3, \ldots \}\\ 
	b_k & = & \frac{1}{\pi} \int_{-\pi}^{\pi} \; f(x) \sin(k x) \mathrm{d}x & \qquad k \in \{1,2,3, \ldots \}\\ 
	\frac{a_0}{2} & = & \frac{1}{2 \pi} \int_{-\pi}^{\pi} \; f(x) \mathrm{d}x
\end{array}
\end{align*}
\subsection[Fourierreihe reell mit Periode $T=\nicefrac{2\pi}{\omega}$]{Reelle Darstellung der Fourierreihe mit allgemeiner Periode $T=\nicefrac{2 \pi}{\omega}$}
\begin{align*}
\begin{array}{rcll}
f(t) & = & \frac{a_0}{2} + \sum_{k=1}^{\infty} \; (a_k \cos(k \omega t) + b_k \sin(k \omega t))  & {}\\ 
\mbox{mit} & {} & {} & {}\\
	a_k & = & \frac{2}{T} \int_{-T/2}^{T/2} \; f(t) \cos(k \omega t) \mathrm{d}t & \qquad k \in \{1,2,3, \ldots \}\\ 
	b_k & = & \frac{2}{T} \int_{-T/2}^{T/2} \; f(t) \sin(k \omega t) \mathrm{d}t & \qquad k \in \{1,2,3, \ldots \}\\ 
	\frac{a_0}{2} & = & \frac{1}{T} \int_{-T/2}^{T/2} \; f(t) \mathrm{d}t
\end{array}
\end{align*}
\subsection{Eigenschaften der reellen Fourierreihe}
\begin{itemize}
\item
$f(t)$ achsensymmetrisch zur $y$-Achse (gerade) $\Rightarrow$ $b_k = 0$, für alle $k \in \{1,2,3, \ldots \}$\\ 
$f(t)$ punktsymmetrisch zum Ursprung (ungerade) $\Rightarrow$ $a_k = 0$, für alle $k \in \{0,1,2,3, \ldots \}$\\
$f(t)$ punktsymmetrisch zu einem Punkt auf der $y$-Achse $\Rightarrow$ $a_k = 0$ für alle $k \in \{1,2,3, \ldots \}$, aber $a_0 / 2 \ne 0$. 
\item
	Klingen die \glqq{}langsamsten\grqq{} Fourierkoeffizienten ab wie $\nicefrac{1}{k^2}$, so ist $f(t)$ stetig.\\
	Klingen die \glqq{}langsamsten\grqq{} Fourierkoeffizienten ab wie $\nicefrac{1}{k}$, so hat $f(t)$ Sprungstellen.\\
(Diese Eigenschaften gelten natürlich auch für die komplexe Darstellung der Fourierreihen.)
\end{itemize}

\subsection[Fourierreihe komplex mit Periode $P=2\pi$]{Komplexe Darstellung der Fourierreihe mit Periode $P=2 \pi$}
\begin{align*}
\begin{array}{rcll}
f(x) & = & \sum_{k=-\infty}^{\infty} \; (c_k \cdot \e^{jkx}) & {}\\ 
\mbox{mit} & {} & {} & {}\\
	c_k & = & \frac{1}{2 \pi} \int_{-\pi}^{\pi} \; f(x) \cdot \e^{-jkx} \mathrm{d}x & \qquad k = 0,\pm 1,\pm 2,\pm 3, \ldots\\ 
	c_0 & = & \frac{1}{2 \pi} \int_{-\pi}^{\pi} \; f(x) \mathrm{d}x
\end{array}
\end{align*}
\subsection[Fourierreihe komplex mit Periode $T=\nicefrac{2\pi}{\omega}$]{Komplexe Darstellung der Fourierreihe mit allgemeiner Periode $T=\nicefrac{2 \pi}{\omega}$}
\begin{align*}
\begin{array}{rcll}
f(t) & = & \sum_{k=-\infty}^{\infty} \; (c_k \cdot \e^{jk \omega t}) & {}\\ 
\mbox{mit} & {} & {} & {}\\
	c_k & = & \frac{1}{T} \int_{-T/2}^{T/2} \; f(t) \cdot \e^{-jk \omega t} \mathrm{d}t & \qquad k = 0,\pm 1,\pm 2,\pm 3, \ldots\\ 
	c_0 & = & \frac{1}{T} \int_{-T/2}^{T/2} \; f(t) \mathrm{d}t
\end{array}
\end{align*}

\newpage

\noindent
\subsection[Umrechnung reelle und komplexe Koeffizienten]{Umrechnungsformeln zwischen den reellen und den komplexen Koeffizienten}
\begin{align*}
\begin{array}{rcl}
c_k & = & \frac{1}{2} ( a_k - j b_k)\\
c_{-k} = c^{*}_{k} & = & \frac{1}{2} ( a_k + j b_k)\\
c_0 & = & \frac{a_0}{2}\\
a_k & = & 2 \Re(c_k)\\
b_k & = & -2 \Im(c_k)
\end{array}
\end{align*}
\subsection[Amplituden-, Phasenspektrum, Fourierkoeffizienten]{Beziehungen zwischen Amplituden-, Phasenspektrum und den Fourierkoeffizienten}

\begin{align*}
\begin{array}{rcl}
A_k & = & \sqrt{a_{k}^{2} + b_{k}^{2}}\\
\tan(\varphi_k) & = & - \frac{b_k}{a_k}\\
|c_k| = |c_{-k}| & = & \frac{1}{2} A_k\\
\varphi_k & = & \arg( c_k)
\end{array}
\end{align*}


\section{Regeln zur Fouriertransformation}
\subsection{Fouriertransformation}
\begin{align*}
\begin{array}{lrcl}
	S(f) & = & \int_{-\infty}^{\infty} \, s(t) \cdot \e^{-j 2 \pi ft} \mathrm{d}t\\ 
	s(t) & = & \int_{-\infty}^{\infty} \, S(f) \cdot \e^{j 2 \pi ft} \mathrm{d}f\\ 
\end{array}
\end{align*}

\subsection{Rechenregeln Fouriertransformation}

\begin{align*}
\begin{array}{lrcl}
	\mbox{} & s(t) & \laplace & S(f)\\
\mbox{Linearität:} & a_1 s_1(t) + a_2 s_2(t) & \laplace & a_1 S_1(f) + a_2 S_2(f) \\
\mbox{Zeitverschiebungssatz:} & s(t-t_0) & \laplace & S(f) \cdot \e^{-j 2 \pi f t_0}\\
\mbox{Vertauschungssatz:} & S(t) & \laplace & s(-f) \\
\mbox{Symmetrie:} & s(-t) & \laplace & S(-f) \\
\mbox{Frequenzverschiebungssatz:} & s(t) \cdot \e^{j 2 \pi f_0 t}  & \laplace & S(f-f_0) \\
\mbox{Ähnlichkeitssatz:} & s(at) & \laplace & \frac{1}{|a|} \cdot S(\frac{f}{a})\\
\mbox{Differentiation im Zeitbereich:} & \frac{d^n s}{\mathrm{d}t^n} & \laplace & (j 2 \pi f)^n \cdot S(f)\\
\mbox{Differentiation im Frequenzbereich:} & (-j 2 \pi t)^n \cdot s(t) & \laplace & \frac{d^n S}{\mathrm{d}f^n}\\
\mbox{Multiplikationssatz:} & t^n \cdot s(t) & \laplace & \frac{j^n}{(2 \pi)^n} \cdot \frac{d^n S}{\mathrm{d}f^n}\\
\mbox{1. Faltungssatz:} & s_1(t) \star s_2(t) & \laplace & S_1(f) \cdot S_2(f)\\
\mbox{2. Faltungssatz:} & s_1(t) \cdot s_2(t) & \laplace & S_1(f) \star S_2(f)\\
\mbox{Integrationssatz:} & \int_{-\infty}^t s(\tau) d \tau & \laplace & \frac{1}{j 2 \pi f} \cdot S(f) + \frac{1}{2} S(0) \cdot \delta(f)\\
\mbox{Parsevalsche Gleichung:} & \int_{-\infty}^{\infty} s^2(t) \mathrm{d}t & = & \int_{-\infty}^{\infty} |S(f)|^2 \mathrm{d}f\\
\end{array}
\end{align*}

\subsection{Korrespondenzen Fouriertransformation}
\begin{align*}
\begin{array}{rcl}
\delta(t) & \laplace & 1\\
1 & \laplace & \delta(f)\\
\sigma(t) & \laplace & \frac{1}{j 2 \pi f} + \frac{1}{2} \delta(f)\\
\end{array}
\end{align*}

\subsection{Eigenschaften der Spektraldichte $S(f)$}
$\mbox{}$\\
Die Spektraldichte $S(f)$ einer geraden Zeitfunktion $s(t)$ ist reell und gerade.\\
Die Spektraldichte $S(f)$ einer ungeraden Zeitfunktion $s(t)$ ist rein imaginär; der Imaginärteil $I(f)$ dieser Spektraldichte ist ungerade.
$\mbox{}$\\

\subsection[Fourierkoeffizienten, Fourierintegral, periodische Funktion]{Beziehung zwischen Fourierkoeffizienten und Fourierintegral beim Übergang zu einer periodischen Funktion}
\begin{align*}
\begin{array}{rcl}
c_k & = & \frac{1}{T} \cdot S_0(k \cdot f_0)\\
	c_0 & = & \frac{1}{T} \int_{-T/2}^{T/2} \, f(t) \mathrm{d}t
\end{array}
\end{align*}


\subsection{Faltung}
\subsubsection{Definition}
\begin{equation}
	f(t) * g(t) = \int_{-\infty}^{\infty}\!f(\tau) \cdot g(t-\tau) \, \mathrm{d}\tau 
\end{equation}

\clearpage

\subsubsection{Berechnung}
\begin{table}[h]
	\begin{tabular}{ll}
		Allgemeines Vorgehen & Beispiel $f(t) * g(t) $\\
				     &$f(t)=\sigma(t)\cdot\e^{-t}$\\
		       & $g(t) = \sigma(t+1) - \sigma(t-1)$\\
		\toprule
		1. Reihenfolge festlegen (mehr $\sigma \Rightarrow$ zuerst) & $g(t)*f(t)$\\
									    &\\
		2. Definition der Faltung übernehmen & $\int_{-\infty}^{\infty}\!g(\tau)\cdot f(t-\tau)\,\mathrm{d}\tau$\\
						      &\\
		3. Fallunterscheidung nach $\tau$ (wann werden die $\sigma=1$?) & $0 < \tau < 1$ \&  $\tau < t$\\
										&\\
		4. Fallunterscheidung nach t & $\begin{cases} 	1. & t < 0 \Rightarrow \int\dots = 0\\ 
								2. & t \in \left[0;1\right] \Rightarrow \int_0^t\e^{-(t-\tau)}\mathrm{d}\tau\\
								3. & t > 1 \Rightarrow \int_0^1\e^{-(t-\tau)}\mathrm{d}\tau
						\end{cases}$\\
						&\\
		5. Integration & Integralbildung\\
							   &\\
		6. Zusammenfassung & $f(t) * g(t) = \begin{cases}
							0, & t < 0\\
						1-\e^{-t}, & 0 < t < 1\\
						\e^{-t}(e-1), & t > 1
						\end{cases}$\\
	\end{tabular}
	\caption{Vorgehen Faltung}
	\label{tab:vorgehen_faltung}
\end{table}

\clearpage

